
%----------------------------------------------------------------------------------------
%	PACKAGES AND OTHER DOCUMENT CONFIGURATIONS
%----------------------------------------------------------------------------------------

\documentclass[a4paper,12pt]{article}

\input{structure.tex} % Include the file specifying the document structure and custom commands


%----------------------------------------------------------------------------------------
%	ASSIGNMENT INFORMATION
%----------------------------------------------------------------------------------------



\newlength{\drop}

\begin{document}
  \begin{titlepage}
    \drop=0.1\textheight
    \centering
   % \vspace*{\baselineskip}
    %\rule{\textwidth}{1.6pt}\vspace*{-\baselineskip}\vspace*{2pt}
    %\rule{\textwidth}{0.4pt}\\[0.5cm]
    \begin{figure}[h]
\centering
\includegraphics[width=4.5cm]{immagini/imgproj}
\end{figure}
    \textbf{Università degli studi di Bergamo}
    \rule{\textwidth}{0.5pt}\\[0.2cm]
   % \rule{\textwidth}{1.6pt}
   SCUOLA DI INGEGNERIA\\[0.15cm]Corso di Laurea Magistrale in Ingegneria Informatica
   \\[\baselineskip]
    {\scshape
    \vspace*{5.5cm}
    \textbf{\LARGE {Relazione progetto C++\\[0.3cm]}}
    {\Large Corso di Programmazione Avanzata\\[2cm] \par}
    {\large{\itshape Giulia Allievi \\ Matricola: 1058231\par}}
    \vfill
    }
    \rule{\textwidth}{0.5pt}\\[0.5cm]
    
   \normalsize \textbf{Anno Accademico 2021-2022}
  \end{titlepage}

%% APPUNTI
% È
% \begin{figure}[h]
% \centering
% \includegraphics[width=16cm]{immagini/img1}
% \caption{Descr ~\cite{rif1}.}
%  \end{figure}
{\linespread{1.3}\selectfont
%----------------------------------------------------------------------------------------
%	PAGINA BIANCA
%----------------------------------------------------------------------------------------
\newpage
\null
\thispagestyle{empty}
\newpage


%----------------------------------------------------------------------------------------
%	INTRODUCTION	
%----------------------------------------------------------------------------------------
\section*{Introduzione} 
L’applicazione progettata permette di visualizzare e gestire le informazioni relative a delle visite mediche, mediante l’inserimento di alcune informazioni riguardanti l’esame, il paziente sul quale questo viene effettuato e il medico che lo esegue. 


%----------------------------------------------------------------------------------------
%	CAPITOLO 1
%----------------------------------------------------------------------------------------
%\clearpage
%\newpage
\section*{Funzionamento} 
L’applicazione è formata da 8 classi: Persona, Dottore, Paziente, Esame, Pet, Mr, PetMr e Metodi (una classe di gestione). Una persona può essere o un paziente o un dottore, entrambi ereditano dalla classe persona in modo pubblico. Entrambe le classi sono caratterizzate da un ID, la classe Dottore possiede un campo specializzazione mentre la classe Paziente un campo per memorizzare la categoria alla quale appartiene. La categoria può assumere uno fra quattro valori possibili, che sono bambino, adulto, anziano e sconosciuto. Le classi di Paziente e Dottore ereditano i campi di nome, cognome, codice fiscale e anno di nascita dalla classe Persona. Le operazioni che si possono svolgere con questi oggetti sono l'inserimento, la visualizzazione e l'ordinamento secondo alcuni criteri. \\
Gli esami possono essere di quattro diversi tipi: un esame ``generico'', Pet, Mr e Pet+Mr, che un esame in cui vengono effettuati in successione un esame Pet e un esame Mr. Ogni esame è caratterizzato da un numero progressivo che lo identifica, dal dottore che lo esegue e dal paziente che si sottopone a quell'esame. Un esame Pet possiede un campo per memorizzare la sua durata, così come per gli esami Mr, ma quest'ultimi hanno anche un campo aggiuntivo per memorizzare l'intensità. Gli esami di tipo Pet+Mr, dato che sono un ``ibrido'' dei due esami precedenti, possiederanno entrambi i campi.  \\
La classe Metodi raccoglie invece tutte le funzionalità che che vengono offerte dal programma e che sono poi richiamate nel main del programma. Nel main è implementato il menù: l'utente digita la lettera corrispondente all’operazione che intende eseguire, sarà quindi richiamato il metodo corrispondente a quella determinata azione. 

%----------------------------------------------------------------------------------------
%	CAPITOLO 2
%----------------------------------------------------------------------------------------
\clearpage
\newpage
\section*{Diagramma delle classi}
Di seguito si riportano i diagrammi delle classi dell’applicazione. \\
- Gerarchia degli utenti (classi Persona, Paziente e Dottore):\\
\begin{figure}[h]
 \centering
 \includegraphics[width=\columnwidth]{diagrammiClassi/gerarchiaPersone}
 \caption{Diagramma delle classi - Gerarchia utenti.}
\end{figure}
\\
\newpage
- Gerarchia delle visite mediche (classi Esame, Pet, Mr e PetMr):\\
\begin{figure}[h]
 \centering
 \includegraphics[width=\columnwidth]{diagrammiClassi/gerarchiaEsami}
 \caption{Diagramma delle classi - Gerarchia esami.}
\end{figure}
\\
\newpage
- Interfaccia dei metodi disponibili:
\begin{figure}[h]
 \centering
 \includegraphics[height=0.73\textheight]{diagrammiClassi/metodi}
 \caption{Diagramma delle classi - Metodi resi disponibili nel programma principale.}
\end{figure}


%----------------------------------------------------------------------------------------
%	CAPITOLO 3
%----------------------------------------------------------------------------------------
\clearpage
\newpage
\section*{Scelte implementative} 
I costruttori sono stati dichiarati protected, in modo tale che possano essere richiamati solo dalle classi derivate. Tutte le classi di Utenti e Visite sono friend della classe Metodi, in modo tale che la classe di gestione possa accedere a tutti i campi e a tutti i metodi di cui necessita. In questa classe inoltre i metodi che si occupano della creazione di un nuovo oggetto sono stati dichiarati private, così che nel main non possano essere richiamati in modo illecito. Anche i metodi utilizzati per il controllo del valore delle variabili sono stati dichiarati private, mentre tutti gli altri metodi sono public. \\
La classe Metodi ha il costruttore privato perché mette a disposizione una singola istanza che verrà poi utilizzata nel main (design pattern \textit{Singleton}). \textbf{ricordati di farlo!!!}\\
Per ogni altra classe invece esiste più di un costruttore (\textit{overloading} degli operatori): se non vengono passati i parametri vengono assegnati dei valori di default, in caso contrario si assegnano i parametri passati dall'utente. L'overloading viene anche sfruttato nella classe Metodi, nei metodi di creazione di un nuovo oggetto da inserire. \\
altro
\\X costruttore e distruttore \\
X campi pubblici e privati \\
- membri virtual e non \\
X overloading \\
- ereditarietà multipla \\
- STL (puntatori safe, algoritmi sort, vector)
classi friend\\
 I campi nome e cognome sono protected, mentre i campi relativi al codice fiscale e all’anno di nascita sono public. \\
I metodi che richiamano i costruttori di una classe sono private così utente non può creare a caso e così come metodi di controllo
La gerarchia delle classi che descrivono gli esami formano un diamante, perché sia la classe Pet che la classe Mr hanno come classe base Esame, la classe PetMr invece rappresenta un esame ibrido fra i due precedenti, questa classe eredita sia dalla classe Pet che dalla classe Mr (mediante il meccanismo dell’ereditarietà multipla del C++). 






%----------------------------------------------------------------------------------------

}
\end{document}
